\documentclass[10pt]{beamer}
\usepackage{fancyvrb}
\usepackage{times}
\usepackage{pgf,pgfarrows,pgfnodes,pgfautomata,pgfheaps}
\usepackage{amsmath,amssymb}
\usepackage[utf8]{inputenc}
\usepackage{colortbl}
\usepackage[portuges]{babel}
\usepackage{graphicx}
\usepackage{color}
\usepackage{hyperref}
\usepackage{pxfonts,txfonts}
\usepackage{url}
\usepackage{underscore}
\usepackage{amsmath}

\usefonttheme{structurebold}

\definecolor{DarkGreen}{rgb}{0,.5,0}
\definecolor{DarkRed}{rgb}{.5,0,0}
\definecolor{DarkBlue}{rgb}{0,0,.5}

\newcommand{\x}{\textbf{\textcolor{DarkGreen}{$\surd$}}}
\newcommand{\xx}{\textbf{\textcolor{DarkBlue}{$\odot$}}}
\newcommand{\xxx}{\textbf{\textcolor{DarkRed}{$\times$}}}

\usepackage{listings}
\renewcommand{\lstlistingname}{Programa}

\lstset{
  numbers=left,
    stepnumber=1,
    firstnumber=1,
    numberstyle=\tiny,
    extendedchars=true,
    breaklines=true,
    frame=tb,
    basicstyle=\footnotesize,
    stringstyle=\ttfamily,
    showstringspaces=false
}
\usetheme{Amsterdam}

\usecolortheme{default}

\setbeamertemplate{navigation symbols}{}

\pgfdeclareimage[height=.5cm]{logo}{imgs/lups_oficial.png}

\logo{\pgfuseimage{logo}}

\setbeameroption{hide notes}
\setbeameroption{hide notes}

\title[]{Um Estudo sobre Ferramentas de Monitoramento de Nuvens}
\author[]{por \\~\\ 
  \textbf{Julio Machado, João Vítor Oliveira, Vitor Alano, \\ Maurício Pilla e Laércio Pilla} \\
    Universidade Federal de Pelotas \\
    Ciência da Computação \\
    \small{Centro de Desenvolvimento Tecnológico}\
    ERAD 2018\\
}

\date{Abril -- 2018  -- Porto Alegre/RS}

\begin{document}

\frame[label=titlepage]{
  \titlepage
}

\frame {
  \frametitle{Sumário}
  \tableofcontents
}

\section{Introdução} \label{sec:introducao}

\frame {
  \frametitle{Introdução}
  \begin{figure} \label{sec:introducao;fig:nuvem}
  \centering \includegraphics[scale=.35]{imgs/data-center-ias.png}
  \end{figure}
}

\section{Monitoramento} \label{sec:monitoramento}

\frame {
  \frametitle{Monitoramento}
  \begin{alertblock}{Relevância em uma nuvem}
  \begin{itemize}
  \item Permite a coleta e visualização de informações físicas;
  \item Auxilia no diagnóstico de problemas sutis;
  \item Provê transparência do desempenho de uma máquina.
    \end{itemize}
  \end{alertblock}
  \begin{exampleblock}{Ferramental}
  \begin{itemize}
  \item Muitas soluções no mercado, 5 foram selecionadas;
  \item Preços variados;
  \item Possuem funcionalidades em comum;
  \end{itemize}
  \end{exampleblock}
}

\section{Ferramentas de Monitoramento} \label{sec:ferramentas_de_monitoramento}

\frame {
  \frametitle{Ferramentas de Monitoramento}
  \begin{block}{}
  \begin{itemize}
  \item Datadog
    \item Logic Monitor
    \item AppDynamics
    \item New Relic
    \item Ganglia
    \end{itemize}
  \end{block}
}

\frame {
  \frametitle{Datadog}
  \begin{exampleblock}{Objetivo}
  \begin{itemize}
  \item Monitoramento de aplicações de alto desempenho.
    \end{itemize}
  \end{exampleblock}
  \begin{block}{Características}
  \begin{itemize}
  \item Open-source;
  \item Documentação mantida pela comunidade;
  \item Interface web.
    \end{itemize}
  \end{block}
}

\frame {
  \frametitle{Logic Monitor}
  \begin{exampleblock}{Objetivo}
  \begin{itemize}
  \item Prover monitoramento de aplicações multicamada (SaaS,PaaS,IaaS);
  \item Prover mecanismos de segurança e controle de accesso.
    \end{itemize}
  \end{exampleblock}
  \begin{block}{Características}
  \begin{itemize}
  \item SaaS;
  \item Closed-source;
  \item Foco em segurança e confiabilidade.
    \end{itemize}
  \end{block}
}

\frame {
  \frametitle{AppDynamics}
  \begin{exampleblock}{Objetivo}
  \begin{itemize}
  \item Monitomento de aplicações de alto desempenho;
  \item Design modular e configurável.
    \end{itemize}
  \end{exampleblock}
  \begin{block}{Características}
  \begin{itemize}
  \item Solução pertencente a Cisco;
  \item Closed-source;
  \item Disponível apenas para Linux e Windows;
  \item 4 áreas de atuação, end-user, application, business e infrastructure.
    \end{itemize}
  \end{block}
}

\frame {
  \frametitle{NewRelic}
  \begin{alertblock}{Objetivo}
  \begin{itemize}
  \item Dinamicidade e flexibilidade;
  \item Coleta de dados em tempo real;
  \item Alertas.
    \end{itemize}
  \end{alertblock}
  \begin{block}{Características}
  \begin{itemize}
  \item SaaS;
  \item Possui templates de regras;
  \item Armazenamento de dados por 7, 30, 90 dias;
  \item Preço é definido pelo tamanho da máquina sendo monitorada.
    \end{itemize}
  \end{block}
}

\frame {
  \frametitle{Gangliga}
  \begin{alertblock}{Objetivo}
  \begin{itemize}
  \item Monitoramento de aplicações distribuídas, como \textit{clusters} e \textit{grids}; 
    \end{itemize}
  \end{alertblock}
  \begin{block}{Características}
  \begin{itemize}
  \item Nasceu na universidade de Berkeley;
  \item Open-source, licença BSD;
  \item Exportação de dados via xml;
  \item XDR para compactação e transferência;
  \item RRDTool para armazenamento e visualização;
    \end{itemize}
  \end{block}
}

\frame {
  \frametitle{Comparação}
  \begin{table}[!ht]
    \begin{tabular}{c c c }
  Monitores & Open Source & Principal Funcionalidade   \\ \hline
    Datadog      &   Sim & Aplicações de alta performace\\\hline
    Logic Monitor      &   Não & Controle e Segurança \\\hline
    AppDynamics      &   Não & Aplicações de alto desempenho\\\hline
    New Relic      &   Não & Coleta de dados em tempo real \\\hline
    Ganglia      &   Sim & Alta performace (Clusters e Grids) \\
    \end{tabular}
  \caption{Tabela Comparativa dos Monitores.}
  \end{table}
}

\section{Conclusão}

\frame {
  \frametitle{Conclusão}
  \begin{block}{Conclusão}
  \begin{itemize}
  \item O monitoramento da Nuvem está envolvido em praticamente todas as tarefas que caracterizam a Computação em Nuvem.
    \end{itemize}
  \end{block}
  \begin{block}{Continuidade}
  \begin{itemize}
  \item Na continuidade deste trabalho será feita uma análise de consumo de \textit{cpu} e memória destas ferramentas. Em seguida será desenvolvido uma ferramenta de monitoramento de Nuvens de baixo consumo.
    \end{itemize}
  \end{block}
}

\frame {
  \frametitle{Agradecimentos}
  \begin{itemize}
  \item Grupo de Pesquisa: LUPS
    \end{itemize}
  \begin{block}{Mais informações:}
  \textbf{E-mail:} \texttt{jmdsneto@inf.ufpel.edu.br}
  \end{block}
}
\end{document}

