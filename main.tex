\documentclass[10pt]{beamer}
\usepackage{fancyvrb}
\usepackage{times}
\usepackage{pgf,pgfarrows,pgfnodes,pgfautomata,pgfheaps}
\usepackage{amsmath,amssymb}
\usepackage[utf8]{inputenc}
\usepackage{colortbl}
\usepackage[portuges]{babel}
\usepackage{graphicx}
\usepackage{color}
\usepackage{hyperref}
\usepackage{pxfonts,txfonts}
\usepackage{url}
\usepackage{underscore}

\usefonttheme{structurebold}

%==================================================================================
\definecolor{DarkGreen}{rgb}{0,.5,0}
\definecolor{DarkRed}{rgb}{.5,0,0}
\definecolor{DarkBlue}{rgb}{0,0,.5}

\newcommand{\x}{\textbf{\textcolor{DarkGreen}{$\surd$}}}
\newcommand{\xx}{\textbf{\textcolor{DarkBlue}{$\odot$}}}
\newcommand{\xxx}{\textbf{\textcolor{DarkRed}{$\times$}}}
%==================================================================================

\usepackage{listings}
\renewcommand{\lstlistingname}{Programa}

\lstset{
numbers=left,
stepnumber=1,
firstnumber=1,
numberstyle=\tiny,
extendedchars=true,
breaklines=true,
frame=tb,
basicstyle=\footnotesize,
stringstyle=\ttfamily,
showstringspaces=false
}

%==================================================================================

\usetheme{Amsterdam}

%==================================================================================

\usecolortheme{default}
\setbeamertemplate{navigation symbols}{}

% inserir logotipo na apresentação
\pgfdeclareimage[height=.5cm]{logo}{imagens/lups_oficial.png}
\logo{\pgfuseimage{logo}}
\setbeameroption{hide notes}
% \logo{%
%     \includegraphics[width=1cm,height=1cm,keepaspectratio]{img/FAPERGS.jpg}
%     \hspace{9.9cm}
%     \pgfuseimage{logo}
%     \hspace{0.1cm}
% }
% \pgfdeclareimage[height=1.5cm]{capes}{img/CAPES.png}
% \capes{%
%     \pgfuseimage{capes}
%     \hspace{9.9cm}
%     \includegraphics[width=1cm,height=1cm,keepaspectratio]{img/CAPES.png}
%     \hspace{1.1cm}
% }
\setbeameroption{hide notes}
%==================================================================================

\title[]{Um Estudo sobre Ferramentas de Monitoramento de Nuvens}
\author[]{por \\~\\ 
\textbf{Julio Machado, \underline{João Vitor}, Vitor Alano, \\ 
 Maurício Pilla e Laércio Pilla} \\
Universidade Federal de Pelotas\\
Ciência da Computação\\
\small{Centro de Desenvolvimento Tecnológico}\
ERAD 2018\\
}

\date{Abril -- 2018  -- Porto Alegre/RS}

%==================================================================================

\begin{document}
\frame[label=titlepage]{\titlepage \vspace{-1cm} 
}

%==================================================================================

\frame{
\frametitle{Sumário}
\tableofcontents
}%==================================================================================

\section{Introdução}

%===================================================================================

%1
\frame{
\frametitle{Introdução}

	\begin{figure}
    		\centering \includegraphics[scale=.3]{imagens/data-center.png}
		\caption { Data center }
 	\end{figure}


}
%2
\frame{
\frametitle{Introdução}

	\begin{figure}
    		\centering \includegraphics[scale=.3]{imagens/data-center-vms.png}
		\caption { Data center com máquinas virtuais }
 	\end{figure}


}

%3
\frame{
\frametitle{Introdução}

	\begin{figure}
    		\centering \includegraphics[scale=.3]{imagens/data-center-ias.png}
		\caption { Nuvem }
 	\end{figure}
}


\section{Monitoramento}
\frame{
\frametitle{Monitoramento}

\begin{block}{Importância do Monitoramento:}
Por um lado, permite o controle de como o \textit{hardware} e \textit{software} estão sendo utilizados. Por outro lado, provê informações de desempenho e indicações de possíveis comportamentos em plataformas e aplicações.    
\end{block}
}

\frame{
\frametitle{Monitoramento}

\begin{block}{Ferramentas:}
    \begin{itemize}
        \item DataDog
        \item Logic Monitor
        \item AppDynamics
        \item New Relic
        \item Ganglia
    \end{itemize}

\end{block}
}


\section{Ferramentas de Monitoramento}
\frame{
\frametitle{Ferramentas}

\begin{block}{DataDog:}
É um monitor que tem como principal finalidade o monitoramento escalar de nuvens e o monitoramento de aplicações de alta performace. É \textit{open-source} e a documentação e integração das plataformas é feita pela comunidade. Sua interface permite visualizar os dados de diversas formas.
\end{block}
}

\frame{
\frametitle{Ferramentas}
\begin{block}{Logic Monitor:}
É um monitor SaaS de código fechado utilizado por empresas como: Adidas, Siemens, Sophos e outros. O foco do Logic Monitor é prover não só o monitoramento das plataformas em camadas, permitindo o controle de fácil acesso, como também em segurança.
\end{block}
}

\frame{
\frametitle{Ferramentas}
\begin{block}{AppDynamics:}
Tem como principal funcionalidade monitorar aplicações de alto desempenho. É constituído por alguns componentes configuráveis: \texttit{controller}, \texttit{MySQL database}, \textit{events service}, e opcionalmente \textit{end user monitoring} (EUM) \textit{server}. O AppDynamics está disponível para os sistemas Linux e Windows.
\end{block}
}


\frame{
\frametitle{Ferramentas}
\begin{block}{NewRelic:}
É um sistema de monitoramento dinâmico e flexível. Com foco em coleta de dados em tempo real. Assim, um alerta é enviado ao usuário logo que algo pré espeficicado pelo próprio usuário em um filtro der errado.
\end{block}
}

\frame{
\frametitle{Ferramentas}
\begin{block}{Ganglia:}
É um sistema de monitoramento distribuído voltado a computação de alta performace, como \textit{clusters} e \textit{grids}. Utiliza XML para representação dos dados, XDR para compactação e transferência de dados e RRDtool para armazenamento e visualização.
\end{block}
}

%TABELA


\section{Conclusão}


\frame{
\frametitle{Conclusão}
\begin{block}{Conclusão}
    \begin{itemize}
        \item O monitoramento da Nuvem está envolvido em praticamente todas as tarefas que caracterizam a Computação em Nuvem.
    \end{itemize}

\end{block}
\begin{block}{Continuidade}
    \begin{itemize}
        \item Na continuidade deste trabalho será feita uma análise de consumo de \textit{cpu} e memória destas ferramentas. Em seguida será desenvolvido uma ferramenta de monitoramento de Nuvens de baixo consumo.

    \end{itemize}

\end{block}

}




\frame{
\frametitle{Agradecimentos}

\begin{itemize}
    
    \item Grupo de Pesquisa: LUPS
\end{itemize}

\begin{block}{Mais informações:}
    \textbf{E-mail:} \texttt{jmdsneto@inf.ufpel.edu.br}

\end{block}
}
%==================================================================================

\end{document}
%==================================================================================

% \bibitem{norman} E. H. Norman {\em Japan's emergence as a modern
%   state} 1940: International Secretariat, Institute of Pacific
%   Relations.