% \documentclass{beamer}
%\documentclass[10pt]{beamer}

\documentclass[10pt, pdf,xcolor=pdftex,dvipsnames,table]{beamer}

\usepackage[brazil]{babel}
\usepackage[utf8] {inputenc}
\usepackage [T1] {fontenc}

%\usepackage[latin1]{inputenc}
\usepackage{pgfpages}
%\usepackage{fancyvrb}
\usepackage{times}
%\usepackage{pgf,pgfarrows,pgfnodes,pgfautomata,pgfheaps}
\usepackage{amsmath,amssymb}
\usepackage{graphicx}
\usepackage{color}
\usepackage{hyperref}
\usepackage{pxfonts,txfonts}
\usepackage{url}
\usefonttheme{structurebold}
\usepackage{hyphenat}
\usepackage{multicol}
%\usepackage{palatino}

\usepackage{listings} % para inserir codigo fonte
\lstset{extendedchars=true,
breaklines=true,
frame=tb,
basicstyle=\footnotesize,
stringstyle=\ttfamily,
showstringspaces=false
}


%\setbeamercolor{structure}{fg=Red!70!black}
%\setbeamercolor{structure}{bg=White}
%\setbeamercovered{transparent}
%\usetheme{Madrid}

%==================================================================================

% \definecolor{DarkGreen}{rgb}{0,.5,0}
% \definecolor{DarkRed}{rgb}{.5,0,0}
% \definecolor{DarkBlue}{rgb}{0,0,.5}

\renewcommand{\lstlistingname}{Listagem}



\newcommand{\x}{\textbf{\textcolor{Green}{$\surd$}}}
\newcommand{\xx}{\textbf{\textcolor{Blue}{$\odot$}}}
\newcommand{\xxx}{\textbf{\textcolor{Red}{$\times$}}}



%==================================================================================

% mudar a quantidade de slides por pagina
% \pgfpagesuselayout{2 on 1}[a4paper,border shrink=5mm]

% \mode<presentation>
% \mode<trans>
% \mode<handout>

% THEMES INSTALADOS
% \usetheme{default}
% \usetheme{AnnArbor}	% BOM
% \usetheme{Antibes} %BOM
% \usetheme{Bergen}
% \usetheme{Berkeley}	% BOM
% \usetheme{Berlin}	% BOM
% \usetheme{Boadilla}
% \usetheme{boxes}
% \usetheme{CambridgeUS}	% BOM
% \usetheme{Copenhagen}	% BOM * grande
% \usetheme{Darmstadt}	% BOM *
% \usetheme{Dresden}	% BOM ++ grande
% \usetheme{Frankfurt}	% BOM grande
% \usetheme{Goettingen}	% BOM nao
% \usetheme{Hannover} 	% BOM nao
% \usetheme{Ilmenau}	% BOM grande
% \usetheme{JuanLesPins} %nao
% \usetheme{Luebeck} %nao grande
% \usetheme{Madrid}
% \usetheme{Malmoe} %grande
% \usetheme{Marburg}	% BOM nao
% \usetheme{Montpellier} %nao
% \usetheme{PaloAlto} %nao
% \usetheme{Pittsburgh}
% \usetheme{Rochester}
% \usetheme{Singapore}	% BOM nao
% \usetheme{Szeged}	% BOM nao
% \usetheme{Warsaw}	% BOM grande


%==================================================================================
% TEMAS EXTRAS
% DEVEM ESTAR NO DIRETORIO DE TRABALHO
%\usetheme{progressbar}
%\usetheme{lankton-keynote}
\usetheme{Amsterdam}
%Options
%\progressbaroptions{headline=sections,frametitle=normal,titlepage=normal}
%\progressbaroptions{headline=sections,frametitle=normal,titlepage=picture}

%\usecolortheme{progressbar}
% \usefonttheme{progressbar}

%\useoutertheme{progressbar}

% Define os elementos dentros dos frames
%\useinnertheme{progressbar}

%==================================================================================

% THEMES DE CORES

% \usecolortheme{beaver} % azul e vermelho - feio
% \usecolortheme{seahorse} % preto - bacaninha
% \usecolortheme{crane}	% Laranja - feio
%\usecolortheme{dove} % branco e preto - legal
% \usecolortheme{albatross} % fundo azul escuro e fonte amarela
% \usecolortheme{rose} %OK  Fonte verde igual ao padrao
% \usecolortheme{orchid} % parecido com original
% \usecolortheme{beetle} % muito escura
% \usecolortheme{fly} % escura
% \usecolortheme{lily}
%\usecolortheme{seagull}  % CINZA - Legal
% \usecolortheme{sidebartab} % nao
% \usecolortheme{whale}	% *
% \usecolortheme{dolphin} % azul   +-
% \usecolortheme{wolverine} % amarelo e azul
% \usecolortheme{default} % azul e branco

%==================================================================================

% Efeitos:
% \transdissolve %dissolve a lamina anterior;
% \transsplitverticalout % a proxima lamina se abre como uma cortina no sentido horizontal;
% \transblindshorizontal % a lamina anterior converte-se linha a linha.

% Para gerar apenas as páginas sem efeitos de overlay use (bom para imprimir):
% \usepackage[handout]{beamer}

% Para colocar número de páginas no slide:
% \setbeamertemplate{footline}[frame number]

% Para retirar a barra de navegação:
\setbeamertemplate{navigation symbols}{}

% inserir logotipo a apresentação
\pgfdeclareimage[height=1.5cm]{logo}{images/lups_oficial.png}
\logo{\pgfuseimage{logo}}

% Ativa ou desativa as anotações
\setbeameroption{hide notes}
%\setbeameroption{show notes}

%==================================================================================

%EVENTO
\renewcommand{\evento}{XIV Escola Regional de Alto Desempenho do Estado do Rio Grande do Sul}

% TITULO DA APRESENTACAO
\title[]{Um Estudo sobre Ferramentas de Monitoramento de Nuvens}

%Autor
\author{~~~Julio Machado, \underline{João Vitor}, Vitor Alano, \\ 
 Maurício Pilla e Laércio Pilla
}

%%%%%%%%%%%%%%%%%%%%%%%%%%%%%%%%%%%%%%%%
% Instituição
%%%%%%%%%%%%%%%%%%%%%%%%%%%%%%%%%%%%%%%%

\institute{Laboratory of Ubiquitous and Parallel Systems \\Universidade Federal de Pelotas \\
}

%%%%%%%%%%%%%%%%%%%%%%%%%%%%%%%%%%%%%%%%
% Data
% sem data (\date{})
% dia de hoje  \date{\today}
%%%%%%%%%%%%%%%%%%%%%%%%%%%%%%%%%%%%%%%%
\date{Abril de 2018}

%\date{
%Orientador: Prof\textordmasculine. Dr. Maurício Lima Pilla \\ 
%Co-orientador: Prof\textordmasculine. Dr. Adenauer Corrêa Yamin \\ 
%\vspace{3mm} ERAD 2013 \\ 
%	  \vspace{3mm} Março de 2013
%}

\begin{document}

% % % % % % % % % % % % % % % % % % % % % %
\frame{\titlepage}
\pgfdeclareimage[height=0.7cm]{logo}{images/lups_timbre.png}
\logo{\pgfuseimage{logo}}


% % % % % % % % % % % % % % % % % % % % % %
\frame{\tableofcontents}

\section{Introdução} \label{sec:introducao}

\frame {
  \frametitle{Introdução}
  \begin{figure} \label{sec:introducao;fig:nuvem}
  \centering \includegraphics[scale=.35]{images/data-center-ias.png}
  \end{figure}
}

\section{Monitoramento} \label{sec:monitoramento}

\frame {
  \frametitle{Monitoramento}
  \begin{block}{Relevância em uma nuvem}
    \begin{itemize}
      \item Permite a coleta e visualização de informações físicas;
      \item Auxilia no diagnóstico de problemas sutis;
      \item Provê transparência do desempenho de uma máquina.
    \end{itemize}
  \end{block}
  \begin{block}{Ferramental}
    \begin{itemize}
      \item Muitas soluções no mercado, 5 foram selecionadas;
      \item Preços variados;
      \item Possuem funcionalidades em comum;
    \end{itemize}
  \end{block}
}

\frame {
  \frametitle{Ferramentas de Monitoramento}
  \begin{block}{}
    \begin{itemize}
      \item Datadog
      \item Logic Monitor
      \item AppDynamics
      \item New Relic
      \item Ganglia
    \end{itemize}
  \end{block}
}

\section{Ferramentas de Monitoramento} \label{sec:ferramentas_de_monitoramento}

\frame {
  \frametitle{Datadog}
  \begin{block}{Objetivo}
    \begin{itemize}
      \item Monitoramento de aplicações de alta performance.
    \end{itemize}
   É \textit{open-source} e a documentação e integração das plataformas é feita pela comunidade. Sua interface permite visualizar os dados de diversas formas.
  \end{block}
  \begin{block}{Características}
    \begin{itemize}
      \item Open-source;
      \item Documentação mantida pela comunidade;
      \item Interface web.
    \end{itemize}
  \end{block}
}

\frame {
  \frametitle{Logic Monitor}
  \begin{block}{Objetivo}
    \begin{itemize}
      \item Prover monitoramento de aplicações multicamada (SaaS,PaaS,IaaS);
      \item Prover mecanismos de segurança e controle de accesso.
    \end{itemize}
  \end{block}
  \begin{block}{Características}
    \begin{itemize}
      \item Solução SaaS;
      \item Closed-source;
      \item Foco em segurança e confiabilidade.
    \end{itemize}
  \end{block}
}

\frame {
  \frametitle{AppDynamics}
  \begin{block}{Objetivo}
    \begin{itemize}
      \item Monitomento de aplicações de alta performance;
      \item Design modular e configurável.
    \end{itemize}
  \end{block}
  \begin{block}{Características}
    \begin{itemize}
      \item Solução pertencente a Cisco;
      \item Closed-source;
      \item Disponível apenas para Linux e Windows;
      \item 4 áreas de atuação, end-user, application, business e infrastructure.
    \end{itemize}
  \end{block}
}

\frame {
  \frametitle{Ferramentas}
  \begin{block}{NewRelic:}
  É um sistema de monitoramento dinâmico e flexível. Com foco em coleta de dados em tempo real. Assim, um alerta é enviado ao usuário logo que algo pré espeficicado pelo próprio usuário em um filtro der errado.
    \end{block}
}

\frame {
  \frametitle{Ferramentas}
  \begin{block}{Ganglia:}
  É um sistema de monitoramento distribuído voltado a computação de alta performace, como \textit{clusters} e \textit{grids}. Utiliza XML para representação dos dados, XDR para compactação e transferência de dados e RRDtool para armazenamento e visualização.
    \end{block}
}

\frame {
  \frametitle{Comparação}
  \begin{table}[!ht]
    \begin{tabular}{c c c }
  Monitores & Open Source & Principal Funcionalidade   \\ \hline
    Datadog      &   Sim & Aplicações de alta performace\\\hline
    Logic Monitor      &   Não & Controle e Segurança \\\hline
    AppDynamics      &   Não & Aplicações de alto desempenho\\\hline
    New Relic      &   Não & Coleta de dados em tempo real \\\hline
    Ganglia      &   Sim & Alta performace (Clusters e Grids) \\
    \end{tabular}
  \caption{Tabela Comparativa dos Monitores.}
  \end{table}
}

\section{Conclusão}

\frame {
  \frametitle{Conclusão}
  \begin{block}{Conclusão}
  \begin{itemize}
  \item O monitoramento da Nuvem está envolvido em praticamente todas as tarefas que caracterizam a Computação em Nuvem.
    \end{itemize}
  \end{block}
  \begin{block}{Continuidade}
  \begin{itemize}
  \item Na continuidade deste trabalho será feita uma análise de consumo de \textit{cpu} e memória destas ferramentas. Em seguida será desenvolvido uma ferramenta de monitoramento de Nuvens de baixo consumo.
    \end{itemize}
  \end{block}
}

\frame {
  \frametitle{Agradecimentos}
  \begin{itemize}
  \item Grupo de Pesquisa: LUPS
    \end{itemize}
  \begin{block}{Mais informações:}
  \textbf{E-mail:} \texttt{jmdsneto@inf.ufpel.edu.br}
  \end{block}
}
\end{document}

